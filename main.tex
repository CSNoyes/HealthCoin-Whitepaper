\documentclass{article}
\usepackage[T1]{fontenc}
\usepackage{hyperref}
\usepackage{graphicx}
\usepackage{amsmath}
\usepackage{pgfplots}

\usepackage[backend=bibtex]{biblatex}
\addbibresource{Biblio.bib}


\title{BitHealth - A P2P Disease Reporting Platform}
\author{Charles Noyes}
\date{July 2015}

\begin{document}
\maketitle

\begin{abstract}
This paper proposes a new architecture for disease reporting in technologically fragmented regions, underpinned by principles adapted from the Bitcoin and BitTorrent projects. Our proposed platform is built from two co-dependant networks. The first is a decentralized registry of users that allows for quick authentication of a user's purported identity. This is done through the inclusion of 'registrations' (notices of private key allocation to a certain identity) on a distributed hash chain (the 'blockchain'). The second is a network based around the dissemination of information through the use of a Distributed Hash Table (DHT). A 'snapshot' of this DHT is included periodically within the blockchain; however, to prevent bloat that could result from Sybil attacks, individual cryptographic proofs of work are included within each submission and used as a filtering mechanism. In summary, we provide a decentralized platform for users to share generalized information. By leveraging protocols developed and proven by much larger open source projects, we hope to mitigate most of the existing attack vectors that could threaten the security of our network.
\end{abstract}


\maketitle

\section{Introduction}
\par Disease reporting networks are becoming an integral part of the disease threat response system~\cite{Heymann:1998ty}. Recent events have shown the importance of these types of tools in the improvement of infectious disease notification~\cite{Ward:2005vk}, and nearly all evaluations of the integration of these systems have concluded that they are a revolutionary development~\cite{Jajosky:2004ty,Lazarus:2009wz,Overhage:2001vq}. 

\par With the recent spike in mobile phone adoption, many of the previously technologically restricted functionalities of the modern Internet and computing landscape have been made available by mobile phones~\cite{Buford:2009jn}. Almost all implementations of these proposed disease reporting networks are intrinsically reliant on the fast communication and coordination that the Internet provides~\cite{Rolfhamre:2005wy,Krause:2007wn}. There is, however, a significant 

\par The motivations to move to a more decentralized system, in this case, are purely practical. Many architecturally similar projects~\cite{Nakamoto:2008ti,Xu:2010vs,Buterin:2014wo} are more philosophically oriented; the sociopolitical climate has catalyzed the feverish pace of innovation. In our case, however, we aim only to provide a better solution to communicable disease surveillance and we wholeheartedly believe that a decentralized model will provide the best platform on which to do so.




\printbibliography

\end{document}

